\documentclass[11pt]{article}

\usepackage{times}
\usepackage[utf8]{inputenc} % allow utf-8 input
\usepackage[T1]{fontenc}    % use 8-bit T1 fonts
\usepackage{url}            % simple URL typesetting
\usepackage{graphics}
\usepackage{color}
\usepackage{amsfonts}       % blackboard math symbols
\usepackage{amsmath}       % blackboard math symbols
\usepackage{amssymb}

\usepackage{lipsum}

\usepackage{geometry}
\geometry{left=2.8cm,right=2.8cm,top=2.6cm,bottom=2.6cm}
\usepackage{fancyhdr}
\pagestyle{fancy}
\usepackage{hyperref}% should be the last package you include

\newcommand{\theteam}{WayneGradientzky}
\newcommand{\team}[1]{\def\theteam{#1}}


\fancyhead[L]{\theteam}
\fancyhead[R]{\thepage}
\cfoot{}

\setlength{\parindent}{0pt}

\team{Rajinish Aneel Bhatia, Bartol Markovinović, Mohd Khizir Siddiqui}
\title{RL-Course 2025/26: Final Project Report}
\author{\theteam}

\begin{document}
\maketitle

\section{Introduction}\label{sec:intro}

The report presents the implementation of three reinforcement learning algorithms by the \emph{WayneGradientzky} team.
The algorithms were finally evaluated on the Hockey competition hosted by the Martius Lab.
\texttt{Hockey} is a 2D-multi-agent, fully observable environment where two agents compete to score goals by hitting a puck into the opponent's goal.

The algorithms implemented are as follows:
\begin{itemize}
    \item \textbf{Task-Driven Model Predictive Control (TDMPC)} by Bartol Markovinović
    \item \textbf{Twin Delayed Deep Deterministic Policy Gradient (TD3)} \cite{fujimoto2018:TD3} by Rajinish Aneel Bhatia
    \item \textbf{Soft Actor-Critic (SAC)} \cite{HaarnojaAbbeelLevine2018:SAC} by Mohd Khizir Siddiqui
\end{itemize}

All the implementations are available on the GitHub repository \url{https://github.com/BartolMarko/HockeyRL}.
Each team member implemented their code individually, with credit given where due.
The report covers the fundamentals of each algorithm, team member contributions in Section \ref{sec:method}, and the evaluation in Section \ref{sec:eval} on the Hockey environment.
Finally, we conclude with a discussion of the results and potential future work in Section \ref{sec:conclusion}.

\section{Method}\label{sec:method}
\subsection{SAC}

\subsection{TD3}

\subsection{TDMPC}

\section{Evaluation}\label{sec:eval}
\subsection{SAC}

\subsection{TD3}

\subsection{TDMPC}

\section{Discussions and Conclusion}\label{sec:conclusion}

\bibliographystyle{abbrv}
\bibliography{main}

\end{document}
